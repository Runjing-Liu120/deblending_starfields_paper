\section{Reparameterized and REINFORCE Gradients}
\label{sec:reparam_details}

The ELBO objective~\eqref{eq:elbo} is of the form
\begin{align}
    \mathcal{L}(\eta) = \Expect_{q_\eta(z)}[f_\eta(z)].
    \label{eq:gen_elbo}
\end{align}
The parameter $\eta$ is to be optimized, and $z$ is the latent variable. The integrating distribution $q$ and the function $f$ depend on $\eta$.

The REINFORCE estimator~\citep{Williams1992reinforce} is a general-purpose unbiased estimate for the gradient of~\eqref{eq:gen_elbo}.
It is given by
\begin{align}
    g_{\textrm{rf}}(z) = \nabla_\eta f_\eta(z) +
            f_\eta(z)  \nabla_\eta \log q_\eta(z)
    \quad \text{for }
    z\sim q_\eta(z).
\end{align}
The REINFORCE estimate is unbiased for the true gradient:
\begin{align}
    \Expect_{q_\eta(z)}[g_{\textrm{rf}}(z)] &=
    \int q_\eta(z) \nabla_\eta f_\eta(z) \; dz+
    \int q_\eta(z) f_\eta(z)  \nabla_\eta \log q_\eta(z)\; dz \\
    &= \int q_\eta(z) \nabla_\eta f_\eta(z) \; dz+
    \int f_\eta(z) \nabla_\eta q_\eta(z)  \; dz \\
    &= \int \nabla_\eta[q_\eta(z) f_\eta(z)] \; dz \\
    &= \nabla_\eta \int q_\eta(z) f_\eta(z) \; dz
    = \nabla_\eta \Expect_{q_\eta(z)}[f_\eta(z)],
\end{align}
assuming that $f$ is well-behaved so that integration and differentiation can be interchanged.

In many applications,
the REINFORCE estimator has too high variance to be useful. One way to lower the variance is to introduce a control variate $C$
\citep{ranganath2013black}, and estimate the gradient as 
\begin{align}
    g_{\textrm{cv}}(z) = \nabla_\eta f_\eta(z) +
        (f_\eta(z) - C)  \nabla_\eta \log q_\eta(z)
    \quad \text{for }
    z\sim q_\eta(z).
\end{align}
This estimate remains unbiased because the score function 
$\nabla_\eta \log q_\eta(z)$ is zero mean under $q$.

A simple but often effective choice of control variate is to let 
$C$ be a second evaluation of $f_\eta$ at an independently drawn $z'\sim q$:
\begin{align}
    g_{\textrm{cv}}(z) = \nabla_\eta f_\eta(z) +
        (f_\eta(z) - f_\eta(z')  \nabla_\eta \log q_\eta(z)
    \quad \text{for }
    z, z' \overset{\mathrm{iid}} \sim q_\eta.
    \label{eq:control_var}
\end{align}
This estimate is unbiased conditional on $z'$ and hence unconditionally unbiased as well. 
We use this control variate for our experiments involving 
the REINFORCE estimator in Section~\ref{sec:elbo_sleep_compare}. 


Alternatively, the reparameterized gradient~\citep{rezende2014stochastic, kingma2013autoencoding} can be used when there exists some distribution $F$ not involving $\eta$ and a differentiable mapping $h_\eta$ such that
\begin{align}
    w \sim F \implies h_\eta(w) \sim q_\eta.
\end{align}
For example, if $q_{\eta}(z) = \mathcal{N}(z; 0, \eta)$ that is, a Gaussian with zero mean and variance $\eta$, one possibility is to let $F$ be the standard Gaussian and $h_\eta(w) = w \sqrt{\eta}$.
The gradient of $\mathcal{L}(\eta)$ can then be written as
\begin{align}
    \nabla_\eta \Expect_{q_\eta(z)}[f_\eta(z)] &=
        \nabla_\eta \Expect_{w\sim F}[f_\eta(h_\eta(w)] = \Expect_{w\sim F}[\nabla_\eta f_\eta(h_\eta(w)],
\end{align}
again assuming the interchangability of integrals and derivatives.
Unbiased gradients arise from the chain rule: 
\begin{align}
    g_{\textrm{rp}}
    = \nabla_\eta f_\eta(h_\eta(w))
    = \nabla_z f_\eta(z)\Big|_{z = h_\eta(w)}
    \nabla_\eta h_\eta(w) \quad \text{for } w\sim F.
\end{align}

The reparameterized gradient includes gradient information $\nabla_z f_\eta(z)$, while the REINFORCE gradient does not. Taking into account the structure of $f$ through its gradient lowers the variance of reparameterized gradient in comparison to the REINFORCE gradient.
However, if $z$ contains discrete components, there cannot be a differentiable mapping $h_\eta$, and the reparameterization trick will not apply.

\subsection{Gradients for the Empirical Comparison of KL Objectives}
In experiments of Section~\ref{sec:elbo_sleep_compare}, 
we used a combination of reparameterized and REINFORCE plus control variate gradients.
Let $\tilde N$ be the vector of per-tile number of stars (a discrete component) and $y$ be the locations and fluxes (continuous components)
Our variational distribution factorizes, so we write the expectation as
\begin{align}
 \mathcal{L}(\eta) = \Expect_{q_\eta(\tilde N)}\Expect_{q_\eta(y)}[f_\eta(\tilde N, y)].
 \label{eq:elbo_factorized}
\end{align}
We use the REINFORCE estimator with control variate \eqref{eq:control_var} for the outer expectation and the reparameterization trick for the inner expectation. 
We first apply REINFORCE to the outer expectation:
\begin{align}
    \nabla_\eta  &
    \Expect_{q_\eta(\tilde N)}\Expect_{q_\eta(y)}\Big[f_\eta(\tilde N, y)\Big] \\
    &=  \Expect_{q_\eta(\tilde N)}\Big[ \nabla_\eta \log q_\eta(\tilde N) \Expect_{q_\eta(y)}\big[f_\eta(\tilde N, y) - 
    \Expect_{q_\eta(\tilde N)}[f_\eta(\tilde N, y)]\big] +
    \nabla_\eta \Expect_{q_\eta(y)}[f_\eta(\tilde N, y)] \Big]\\
    &\approx \nabla_\eta \log q_\eta(\tilde N) \Expect_{q_\eta(y)}[f_\eta(\tilde N, y) - f_\eta(\tilde M, y)] +
    \nabla_\eta \Expect_{q_\eta(y)}[f_\eta(\tilde N, y)]
    \label{eq:reinforce_partial}
\end{align}
 for $\tilde N,\tilde M \overset{iid}\sim q_\eta$.
Then we use the reparameterization trick for $y$, so
\begin{align}
    \Expect_{q_\eta(y)}[f_\eta(\tilde N, y) - f_\eta( \tilde M, y)] &\approx f_\eta(\tilde N, h_\eta(w)) - f_\eta(\tilde M, h_\eta(w))\\
    \nabla_\eta \Expect_{q_\eta(y)}[f_\eta(\tilde N, y)] &\approx  \nabla_y f_\eta(\tilde N, y)\Big|_{y = h_\eta(w)}
    \nabla_\eta h_\eta(w)
    \label{eq:reparam_partial}
\end{align}
for $w \sim F$, where $h_\eta$ and $F$ are chosen appropriately. Combining~\eqref{eq:reinforce_partial} and~\eqref{eq:reparam_partial}, our gradient estimator is
\begin{align}
    g(z) = \nabla_\eta \log q_\eta(\tilde N)
    [f_\eta(\tilde N, h_\eta(w)) - f_\eta(\tilde M, h_\eta(w))] +
    \nabla_y f_\eta(\tilde N, y)\Big|_{y = h_\eta(w)}
    \nabla_\eta h_\eta(w).
    \label{eq:reiforce_final}
\end{align}
Equation~\eqref{eq:reiforce_final} is what our main text called the ``REINFORCE gradient."
These gradients produced the optimization path in Figure~\ref{fig:optim_path}(a).

The ``reparameterized" gradient in Section~\ref{sec:elbo_sleep_compare} requires integrating out $\tilde N$.
Here, we write the outer expectation in~\eqref{eq:elbo_factorized} as a summation of $4^{N_{max}+1}$ terms
(recall our experiments have four tiles, and at most $N_{max}$ stars per tile), 
with each term representing a different possible 
assignment of the number of stars to each tile: 
\begin{align}
\mathcal{L}(\eta) = \sum_{\tilde n}\Expect_{q_\eta(y)}[f_\eta(\tilde n, y)].
\end{align}
Then, the reparameterization trick is applied to each term of the summation.
Stochastic gradients are computed as,
\begin{align}
g(z) = \sum_{\tilde n=1}\nabla_y f_\eta(\tilde n, y)\Big|_{y = h_\eta(w)}
\nabla_\eta h_\eta(w) \quad \text{for } w\sim F.
\end{align}
Gradients of this form produced the optimization path in Figure~\ref{fig:optim_path}(b).
No REINFORCE estimates were required. 

Notice that to compute these re-parameterized gradients we require 
a summation of $(S \times T)^{N_{max}+1}$ terms 
(recall from the main text that $S \times T$ is the total number 
of tiles in an image). 
For large images, i.e. those requiring many tiles, computing re-parameterized gradients would be infeasible. 

% for $N\sim q_\eta(N)$ and $w\sim F$. Because the variational distribution on locations and fluxes are transformation of Gaussians (recall locations are logit-normal and fluxes log-normal), we take $F$ to be the standard Gaussian, while $h(\cdot ; \eta)$ shifts and scales the standard Gaussian according the the parameters returned by the neural network and applies the appropriate transformation.
