% In mean-field variational inference, the ELBO~\eqref{eq:elbo} is often optimized by coordinate ascent in the parameters of the variational distribution.
% The coordinate ascent updates can be derived in closed form in the special case of exponential family models that are conditionally conjugate~\citep{Blei_2017_vi_review}.

Procedures such as black-box variational inference (BBVI)~\citep{ranganath2013black} and
automatic-differentiation variational inference (ADVI)~\citep{kucukelbir2016automatic}
optimize the ELBO  without the need for
deriving analytic expressions for the expectation over $q_\eta$.
% Examples of such ``automatic" procedures include
%and automatic-differentiation variational inference (ADVI)~\citep{kucukelbir2016automatic}.
These approaches employ stochastic gradient descent (SGD);
they sample latent variables from $q_\eta$ and produce an unbiased estimate for the gradient of the ELBO by taking advantage of modern automatic differentiation tools.
ADVI is closely related to the reparameterization trick \citep{SpallOptimization2003,kingma2013autoencoding, rezende2014stochastic}, which is often used to fit variational autoencoders and applies when the latent variables are continuous.

In our model, the number of stars $N$ is discrete.
The REINFORCE estimator~\citep{Williams1992reinforce} is one way to produce an unbiased stochastic gradient for both continuous and discrete latent variables.
However, REINFORCE gradients often suffer from high variance in practice, even with the introduction of control variates, resulting in slow convergence of stochastic optimization. We find this to be true in our empirical study (Section~\ref{sec:elbo_sleep_compare}).
% See Appendix~\ref{sec:reparam_details} for details concerning both the reparameterization and REINFORCE estimators.

The key difficulty in constructing stochastic gradients of the ELBO is that the
integrating distribution depends on the optimization parameter $\eta$.
We instead maximize a negated expectation of the ``forward" KL divergence:
\begin{align}
    \mathcal{L}_{fwd}(\eta) :=
    - \Expect_{x \sim p(x)}\Big[\KL(p(z \mid x) \| q_\eta(z \mid x))\Big]
    \label{eq:sleep_obj},
\end{align}
an objective that appears in the ``sleep phase" of the wake-sleep algorithm
\citep{Hinton1995wake_sleep, bornschein2014reweighted,le2020revisiting}
and was re-introduced by \cite{ambrogioni2019favi}, who called this approach ``Forward Amortized Variational Inference."

Section~\ref{sec:sleep_details} details a simple gradient estimator for~\eqref{eq:sleep_obj} that does not require reparameterization or REINFORCE.

There are two key differences between the expected forward KL objective~\eqref{eq:sleep_obj}
and the ELBO~\eqref{eq:elbo}.
First, recall that maximizing the ELBO is equivalent to minimizing
$\KL(q\|p)$; the KL in $\mathcal{L}_{fwd}$ transposes the arguments.
Second, the outer expectation over $p(x)$ in $\mathcal{L}_{fwd}$ gives a different meaning to the objective.
The ELBO objective seeks $\eta$ to minimize the $\KL$ between $q_\eta(z \mid x)$ and $p(z \mid x)$
for fixed, observed data $x$,
in this case the $H\times W$ image.
In contrast, minimizing $\mathcal{L}_{fwd}$ minimizes the $\KL$
on average over all possible data $x$,
as weighted by $p(x)$.
The target is no longer an approximate posterior for the observed data, but rather
an approximate posterior that is ``good on average" over all possible data under the model $p(x)$.

% Therefore, it is imperative that the model $p(x)$ approximates the true underlying data-generating mechanism well.
% Thus, the wake-sleep algorithm also incorporates a ``wake phase" to estimate model parameters.
% In our application, these model parameters include PSF parameters $\pi$ and background parameters $\beta$. Define
% $\phi:=(\pi, \beta)$, and denote the dependence of the generative model on $\phi$ using subscripts, $p_\phi$.
%
% To estimate model parameters, one would ideally optimize the marginal log-likelihood $\log p_\phi(x)$.
% However, since $\log p_\phi(x)$ is intractable, the wake-phase optimizes for $\phi$ using the ELBO~\eqref{eq:elbo}
% as a surrogate for the intractable log-likelihood.
% The ELBO is a lower bound of
% $\log p_\phi(x)$;
% it is equal to $\log p_\phi(x)$ when $q_\eta(z \mid x)$
% = $p_\phi(z \mid x)$.
%
% The {\itshape wake-sleep} algorithm alternates between the two objectives:
% \begin{align}
%     \text{{\bf Sleep phase: }} &
%     \eta_{t} = \argmax_{\eta} -\Expect_{x \sim p_\phi(x)}\Big[\KL(p_{\phi_{t-1}}(z \mid x) \| q_\eta(z \mid x))\Big]
%     \label{eq:sleep_phase_summary};
%     \\
%     \text{{\bf Wake phase: }} & \phi_{t} = \argmax_{\phi}\; \Expect_{q_{\eta_t}(z \mid x)}\Big[\log p_{\phi}(x, z) - \log q_{\eta_t}(z \mid x)\Big],
%     \label{eq:wake_phase_summary}
% \end{align}
% for iterations $t = 1, ..., T$.
%
% Stochastic gradients of the expectation in the wake-phase are simple to compute. Because the integrating distribution does not depend on the optimization parameter $\phi$ in the wake phase, unbiased stochastic gradients are simply computed as
% \begin{align}
%     \nabla_\phi \log p_\phi(x, z) \quad \text{ for } z\sim q_\eta.
%     \label{eq:mstep_grad}
% \end{align}
% Section~\ref{sec:sleep_details} shows that a similarly simple gradient estimator exists for the sleep objective.
%

% While the E-step~\eqref{eq:e_step} is not conducive to simple gradient estimators, we will see in Section~\ref{sec:sleep_details} that the sleep phase objective
% results in a similarly straightforward gradient estimator as in~\eqref{eq:mstep_grad}.
% In Section~\ref{sec:estep_sleep_compare},
% we compare optimizing the E-step with optimizing the sleep phase. In our application of localizing stars,
% the ELBO objective in the E-step suffers from poor local optima where the variational distribution is far from the true posterior in KL divergence; the sleep phase objective appears to better avoid these local optima.
