StarNet employs variational inference and outperforms both an MCMC-based probabilistic cataloger and a non-model-based approach in terms of accuracy and runtime. 
Under the framework of probabilistic modeling, 
StarNet produces catalogs in which uncertainties are captured by a Bayesian posterior over the set of all catalogs.
Importantly, unlike MCMC, StarNet also has the capacity to scale probabilistic cataloging to process large astronomical surveys. 

The quality of StarNet detections is the result of optimizing the forward KL, a different objective than the one traditionally used in variational inference. 
Optimizing the forward KL allows the variational posterior to be fit on large amounts of {\itshape complete} data -- the image along with its latent catalog -- generated from the statistical model. 
While labeled data from simulators has been used in other astronomy applications to train deep neural networks (see for example~\cite{Lanusse_2017_cmudeeplens} and \cite{huang2019finding}), StarNet is the first training procedure to employ simulated data in a statistical framework: the neural network specifies an approximate Bayesian posterior. 

This variational approach, unlike previous MCMC approaches, enables StarNet to estimate model parameters such as the PSF and sky background.
While the current work focuses on PSF models, our methodology can be extended to more general sources such as galaxies. 
Unsurprisingly, the current performance of StarNet is sub-optimal for cataloging regions of the sky that contain both stars and galaxies, due to model misfit (Appendix~\ref{sec:results_sparse_field}).  

One promising direction is using a neural network for the generative model and fitting a deep generative model for galaxies~\citep{Regier2015ADG, Reiman_2019_gans_deblend, lanusse2020deep, Arcelin_2020}. 
Here, a neural network encodes a conditional likelihood of galaxy images given a low-dimensional galaxy representation. 
Using a neural network to encode a likelihood extends the flexibility of galaxy models beyond the simple models used here. 


The statistical framework in this research lays the foundation for building flexible models to incorporate the cataloging of all celestial objects. 
Future astronomical surveys will only expand in terms of the volume of data they are able to amass. 
As telescopes peer deeper into space, fields will reveal more sources and images will become more crowded. 
The uncertainties in crowded fields necessitate a probabilistic approach. 
Our method holds the promise of providing  a scalable inference tool that can meet the challenges of these future surveys. 


