\noindent{\bf Evaluating the variational distribution}

% \noindent We can use the mapping~\eqref{eq:patch_to_full_map}
% to sample catalogs from our variational distribution.
% If $\tau$ is the mapping in~\eqref{eq:patch_to_full_map}, then 
% we can express the variational distribution
% on the full image catalog $z$ as 
% \begin{align}
%     z &\stackrel{d}{=} \tau\Big( \big(\tilde N^{(s, t)}, \tilde \ell^{(s, t)}, \tilde f^{(s, t)}\big)_{s=1, t = 1}^{S,T}\Big), \notag \\  
%         &\text{where } 
%         \big(\tilde N^{(s, t)}, \tilde \ell^{(s, t)}, \tilde f^{(s, t)}\big)_{s=1, t = 1}^{S,T}
%         \sim 
%         q_\eta\big( \cdot |x\big).
% \end{align}

\noindent Evaluating
the ELBO requires computing the probability of 
$q_\eta(z | x)$
for any given catalog $z = \{N, (\ell_i, f_{i,1}, ..., f_{i,B})_{i = 1}^N\}$. 
By~\eqref{eq:pull_back_of_q}, 
it suffices to evaluate $\tilde q_\eta(\tau^{-1}(z) | x)$. 

Here, $\tau^{-1}(z)$ is a {\itshape set} of tile latent variables because the mapping from tile latent variables to catalogs $z$ is not injective, as we now explain.  

Locations in the catalog $\{\ell_i\}_{i=1}^N$
determine the number of stars on tile $(s,t)$. 
The number of stars $\tilde N^{(s,t)}$ is simply the count of the locations that reside within that tile:
\begin{align}
\tilde N^{(s,t)} = \sum_{i=1}^N 
\mathbf 1 \Big\{\ell_i\in [Rs, R(s+1)] \times [Rt, R(t+1)]\Big\},
\end{align}
where $\mathbf{1}\{\cdot\}$ is the indicator function, equal to one if true and zero if false.

Now, consider $\tilde\ell^{(s, t)}$ and $\tilde f^{(s, t)}$, the triangular array of locations and fluxes on tile $(s,t)$. 
For each $(s,t)$, the $\tilde N^{(s,t)}$-th row 
of the triangular array of fluxes and locations is
determined by the locations and fluxes of stars imaged in tile $(s,t)$; they are determined by the catalog $z$. However, the other rows 
of the triangular arrays are not determined by 
the catalog $z$; they are free to take any value in their domain. Therefore, the mapping $\tau$ is not injective. 

Thus, evaluating the probability of $\tau^{-1}(z)$ under $\tilde q_\eta$ requires marginalizing over the rows of the triangular arrays $\ell^{(s, t)}$ and $\tilde f^{(s, t)}$ that are not determined by $z$. However, 
because $\tilde q_\eta$ fully factorizes, the terms 
in~\eqref{eq:factorize_within_patch} where $n \not= \tilde N^{(s,t)}$ do not enter the
product
after marginalization.
Applying this observation and combining~\eqref{eq:factorize_within_patch} and ~\eqref{eq:factorize_patches}, $\tilde q(\tau^{-1}(z) | x)$ becomes
\begin{align}
    \tilde q(\tau^{-1}(z) | x) = \prod_{s=1}^S\prod_{t=1}^T
    \Big\{
    \tilde q_\eta(\tilde N^{(s,t)} | x) 
    \prod_{i = 1}^{\tilde N^{(s,t)}}
    \tilde q_\eta\big(\tilde \ell_{\tilde N^{(s,t)},i}^{(s, t)} | x\big)
    \tilde q_\eta\big(\tilde f_{\tilde N^{(s,t)},i}^{(s, t)} | x\big)
    \Big\}.
\end{align}
In words, given a catalog $z$,
first convert $z$ to tile random variables;
to compute $q_\eta(z | x)$, it suffices to evaluate $\tilde q_\eta$ only at the rows of triangular 
arrays determined by the number 
of stars falling in each tile. 





