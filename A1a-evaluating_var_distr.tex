The sleep phase optimization requires evaluating
$q_\eta(z \mid x)$ for a given catalog
\[z = \{N, (\ell_i, f_{i,1}, ..., f_{i,B})_{i = 1}^N\}.\]
By~\eqref{eq:pull_back_of_q}, 
it suffices to evaluate $\tilde q_\eta(\tau^{-1}(z) \mid x)$, 
where $\tau$ is the mapping from tile latent variables $\tilde z$ to the catalog $z$ as described in Section~\ref{sec:factorization}, 
and $\tilde q_\eta$ is the distribution on tile latent variables. 

Here, $\tau^{-1}(z)$ is a {\itshape set} of tile latent variables because the mapping from tile latent variables $\tilde z$ to catalogs $z$ is not injective, as we now explain.  

Locations in the catalog $\{\ell_i\}_{i=1}^N$
determine the number of stars on tile $(s,t)$. 
The number of stars $\tilde N^{(s,t)}$ is simply the count of the locations that reside within that tile:
\begin{align}
\tilde N^{(s,t)} = \sum_{i=1}^N 
\mathbf 1 \Big\{\ell_i\in [Rs, R(s+1)] \times [Rt, R(t+1)]\Big\},
\end{align}
where $\mathbf{1}\{\cdot\}$ is the indicator function, equal to one its predicate is true if true and zero otherwise.

Now, consider $\tilde\ell^{(s, t)}$ and $\tilde f^{(s, t)}$, the triangular array of locations and fluxes on tile $(s,t)$. 
For each $(s,t)$, the $\tilde N^{(s,t)}$-th row 
of the triangular array of fluxes and locations is
determined by the locations and fluxes of stars imaged in tile $(s,t)$; they are determined by the catalog $z$. However, the other rows 
of the triangular arrays are not determined by 
the catalog $z$; they are free to take any value in their domain. Therefore, the mapping $\tau$ is not injective. 

Thus, evaluating the probability of $\tau^{-1}(z)$ under $\tilde q_\eta$ requires marginalizing over the rows of the triangular arrays $\ell^{(s, t)}$ and $\tilde f^{(s, t)}$ that are not determined by $z$. However, 
because $\tilde q_\eta$ fully factorizes, the terms 
where $n \not= \tilde N^{(s,t)}$ do not enter the
product
after marginalization.
On each tile $(x,t)$,
\begin{align}
    \tilde q_\eta\big(\tilde N^{(s, t)}, \tilde \ell^{(s, t)}, \tilde f^{(s, t)} \mid x\big) &= 
    \tilde q_\eta(\tilde N^{(s,t)} \mid x) 
    \prod_{n = 1}^{N_{max}}\prod_{i = 1}^{n}
    \tilde q_\eta\big(\tilde \ell_{n,i}^{(s, t)} \mid x\big)
    \tilde q_\eta\big(\tilde f_{n,i}^{(s, t)} \mid x\big) \\
    &= \tilde q_\eta(\tilde N^{(s,t)} \mid x) 
    \prod_{i = 1}^{\tilde N^{(s,t)}}
    \tilde q_\eta\big(\tilde \ell_{\tilde N^{(s,t)},i}^{(s, t)} \mid x\big)
    \tilde q_\eta\big(\tilde f_{\tilde N^{(s,t)},i}^{(s, t)} \mid x\big).
\end{align}

% Applying this observation and combining~\eqref{eq:factorize_within_patch} and ~\eqref{eq:factorize_patches}, $\tilde q(\tau^{-1}(z) \mid x)$ becomes
% \begin{align}
%     \tilde q(\tau^{-1}(z) \mid x) = \prod_{s=1}^S\prod_{t=1}^T
%     \Big\{
%     \tilde q_\eta(\tilde N^{(s,t)} \mid x) 
%     \prod_{i = 1}^{\tilde N^{(s,t)}}
%     \tilde q_\eta\big(\tilde \ell_{\tilde N^{(s,t)},i}^{(s, t)} \mid x\big)
%     \tilde q_\eta\big(\tilde f_{\tilde N^{(s,t)},i}^{(s, t)} \mid x\big)
%     \Big\}.
% \end{align}
In words, given a catalog $z$,
first convert $z$ to tile latent variables;
then on each tile, it suffices to evaluate $\tilde q_\eta$ only at the rows of triangular 
arrays determined by the number 
of stars falling in each tile. 

The last technical detail is computing 
a given row of a triangular array. 
Because catalogs are sets,
each star in the tile must 
be matched with corresponding variational parameters. 
Let $(\tilde \ell_i, \tilde f_i)_{i = 1}^n$ generically denote the tile latent variables in the $n$-th row of a triangular array, on some tile. 
We find the permutation of the tile latent variables that maximize its log-probability under $q$. 
Recalling the variational distribution in~\eqref{eq:var_distr_loc}-\eqref{eq:var_distr_f}, we permute the stars by finding 
\begin{align}
    \argmax_\pi \Big\{ \prod_{i=1}^n \text{LogitNormal}(\tilde \ell_{\pi(i)}; \mu_{\ell_{i}}, \nu_{\ell_{i}})\times 
	\text{LogNormal}(\tilde f_{\pi(i)}; \mu_{f_{i}}, \sigma^2_{f_{i}})\Big\}
\end{align}
where the argmax is taken over all permutations on $\{1, ..., n\}$. This is feasible because on each tile $N_{max} = 3$, so we only need to search through $3!=6$ possibilities. 