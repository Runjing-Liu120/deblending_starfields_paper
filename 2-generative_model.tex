Stars in the sky radiate photons. A telescope image records the number of photons that arrive at each pixel. Telescope images also consist of multiple filter bands, each absorbing photons within a specified range of wavelengths. 

For a given $H \times W$ pixel image with $B$ filter bands, our goal is to infer a catalog of 
stars. 
The catalog specifies the number of stars 
in the image; then for each star, the catalog 
records its location and its flux, or brightness,
in each band. 
The space of latent variables 
$\mathcal{Z}$ is the collection of all possible catalogs of the form
\begin{align}
    z := \{N, (\ell_i, f_{i,1}, ..., f_{i,B})_{i = 1}^N\},
    \label{eq:cat_formulation}
\end{align}
where the number of stars in the catalog
is $N\in\mathbb{N}$;
% \jeff{Unfortunately I don't think astronomers define catalog this way. They think of a catalog as including uncertainty estimates, not as a realization of some random variables.}
% \bryan{I think this is actually how Portillos refers to a catalog -- he does admit that this isn't 
% what astronomers traditionally think of a catalog. The reason we need this is because our posterior is really defined over the space of all possible catalogs -- each sample from the posterior is a catalog as defined in \eqref{eq:cat_formulation}. I'll try to set this up more explicitly in the introduction. }
% \jeff{Sounds good. I really like defining a catalog as a sample, not a distribution. If Portillos defines it this way too, perhaps we can make the new definition stick.}
the location of the $i$th star is $\ell_i \in \Reals^2$; and 
the flux of the $i$th star in the $b$th band is $f_{i, b}\in\Reals^+$. 

The prior is a distribution over the set of catalogs $\mathcal{Z}$. This work uses a marked spatial Poisson process. First, draw the number of stars contained in the $H\times W$ image as
\begin{align}
	N &\sim \text{Poisson}(\mu HW).
\end{align}
% \jeff{Better to talk about the marked point process later (or earlier, in the introduction). Just describe the model as simply as possible here.}
Next, draw locations
\begin{align}
  \ell_1, ..., \ell_N | N &\stackrel{iid}{\sim} \text{Uniform}([0, H] \times [0, W]). 
 \end{align}
The fluxes in the first band are from a power law distribution:
\begin{align}
    f_{1, 1}, ..., f_{N,1} | N & 
    \stackrel{iid}{\sim} \text{Pareto}(f_{min}, \alpha) 
    \label{eq:flux_prior}.
\end{align}
Fluxes in other bands are described relative to the first band. The relative difference is known as ``color." Colors are drawn as 
% \jeff{We need to define color first; surrounding it with quotes isn't enough.}
% The colors are drawn as
\begin{align}
  c_{1, b}, ..., c_{N,b} | N  & 
      \stackrel{iid}{\sim} \mathcal{N}(\mu_c, \sigma^2_c), \quad b = 2, ..., B.
\end{align}
Given the flux in the first band $f_{i,1}$ and color $c_{i,b}$,
the flux in band $b$ is  $f_{i,b} = f_{i,1} \times 10^{c_{i,b} / 2.5}$.
\jeff{Is the reference band always 1? I usually think of the r band as having index 3. What does Portillo do?}
\bryan{I didn't actually mean the index of the band in the telescope -- just wanted an enumerate of bands $1, ..., B$}
% the flux in the $b$th band $f_{i,b}$ is $f_{i,1} \times 10^{c_{i,b} / 2.5}$.
% \jeff{$f^b$ looks like a value raised to an exponent. $f^{(b)}$ instead?}

% We call the first band a ``reference band'' and define colors with respect to the band. 
% An image with $B$ bands, has $B - 1$ colors.
% The relation between flux in the $b$th band $f_b$ and color $c_{i, b}$ is 
% $c_{i,b} = 2.5\log_10(f_{i,b}/f_{i,1})$

% In our formulation, each value of $N$ corresponds to a disjoint set of $N$ locations and fluxes.
% Let $\ell$ and $f$ (without subscripts) each denote the triangular array of latent variables.

% Having drawn $N$, we index into the $N$th row of triangular array, and use this set of locations and fluxes to construct 
% the image. The expected number of photoelectrons measured at pixel $(h,w)$ in band $b$ is

Having drawn a catalog $z = \{N, (\ell_i, f_{i,1}, ..., f_{i,B})_{i = 1}^N\}$,
the expected number of photoelectrons measured at pixel $(h,w)$ in band $b$ is
\begin{align}
  \lambda^b_{hw} = I^{b}(h, w) + \sum_{i = 1}^N f_{i,b} \mathcal{P}^b\big(h - \ell_{i}[1], w - \ell_{i}[2]\big),
  \label{eq:expected_intensity}
\end{align}
where $I^{b}$ is the background intensity, which vary by pixel and band. $\mathcal{P}^b$ is the point spread function (PSF) for band $b$. The PSF
is a function 
\begin{align}
\mathcal{P}^b : \Reals \times \Reals \mapsto \Reals^+,
\end{align}
describing how a stellar point source appears
on our image. The PSF model is a weighted average between a Gaussian ``core" and a power-law ``wing" as described in~\cite{Xin2018psf}. For each band, the PSF has the form
\jeff{Better to define the parameters before using them.}
\begin{align}
    \mathcal{P}(h, w) = \frac{\exp(\frac{-(h^2 + w^2)}{2\sigma_1^2}) + 
                            \zeta \exp(\frac{-(h^2 + w^2)}{2\sigma_2^2}) + 
                            \rho(1 + \frac{h^2 + w^2}{\gamma\sigma^2_P})^{-\gamma/2} }{1 + \zeta + \rho},
\end{align}
The PSF parameters vary by band. Define 
$\pi := (\sigma_{1}^{(b)}, \sigma_{2}^{(b)}, \sigma_{P}^{(b)}, \gamma^{(b)}, \zeta^{(b)}, \rho^{(b)})_{b=1}^B$ as the collection of PSF parameters across all bands. 

The background intensity at pixel $(h,w)$ is modeled with an affine function: 
\begin{align}
    I^{b}(h,w) = \beta_0^{b} + \beta_1^{b} \times h + \beta_2^{b} \times w.
\end{align}
The background parameters are specific to the band. 

The model parameters $\pi$ for the PSF and $\beta$ are estimated by the SDSS software pipeline and reported along with the release of each SDSS image. Alternatively Section~\ref{sec:model_params} introduces a method to estimate these parameters jointly with the approximate posterior. 

Finally, the likelihood of $x^b_{hw}$, the observed number of photoelectrons at pixel $(h,w)$ and band $b$, is Gaussian
with mean and variance equal to $\lambda^b_{hw}$. 
% This model is reasonable due to the law of rare events and the Gaussian approximation to the Poisson.
% \jeff{If we're actually using a Gaussian rather than a Poisson, it's probably better not to mention the Poisson.
% Just say we model the number as Gaussian with var == mean.}
Thus, the observed pixel intensities are
\begin{align}
  x_{hw}^b | z \overset{ind}{\sim} \mathcal{N}(\lambda^b_{hw}, \lambda^b_{hw}),
  \quad 
  b = 1, ..., B; \;
  h = 1,..., H; \; 
  w = 1, ..., W. 
\end{align}
The scaling of variance with the expected intensity is motivated by the Poissonian nature of photon arrivals. 
% \jeff{Better to use lowercase $x$ rather than $X$ if you're not going to use uppercase for all the random variables}
Explicitly, the log likelihood is
\begin{align}
    \log p(x | z) &= \sum_{b = 1}^{B} \sum_{h = 1}^H \sum_{w = 1}^W 
        \Big\{\frac{1}{2\lambda^b_{hw}}(x_{hw}^b  - \lambda^b_{hw})^2 - 
               \frac{1}{2}\log(2\pi\lambda^b_{hw})\Big\}
    \label{eq:loglik}.
\end{align}

The likelihood and priors described above are similar to those
in previous probabilistic cataloging methods~\cite{Brewer_2013, Portillo_2017, Feder_2019, regier2019_celeste}. 
The only difference between our model and~\cite{Portillo_2017, Feder_2019} is that we use a Poisson prior on $N$, while \cite{Portillo_2017, Feder_2019} use an exponential prior. 

In \cite{Brewer_2013}, a broken power law prior was used for the flux. 
Further hyper-priors were placed on the parameters of the broken power law, and the parameters of the broken power law were treated as latent variables. This would be analogous to placing a hyper-prior on $\alpha$ in \eqref{eq:flux_prior} in our model. 
Alternatively, though not explored in this work, $\alpha$ can be optimized as a model parameter along with the PSF and sky background. 
The same principle applies to the Poisson prior parameter, $\mu$. 



% In catalog M2, we use the same prior parameters as~\cite{Portillo_2017, Feder_2019}: the slope of the power law on fluxes is set at  $\alpha = 2$; the minimum flux $f_{min}$ to corresponds to the SDSS 
% lower detection limit ($\approx$ 22 magnitude); and the color prior has a standard Gaussian prior, which only disfavors extremely atypical colors, $|c_{i,b}| > 3$. 

% One difference is that \cite{Portillo_2017, Feder_2019} use an exponential prior on $N$. We use a Poisson prior and set $\mu = 0.15$, slightly denser than the $0.1$ stars per pixel 
% brighter than 22mag found in the Hubble catalog of M2. Appendix TODO shows that our model is insensitive to the choice of Poisson mean. 

