Most astronomical light sources can be classified as either stars or galaxies. We focus on applications where all sources are well-modeled as stars (examples include images of globular clusters and images of our own Milky Way). 

Stars radiate photons. 
A telescope image records the number of photons that arrive at each pixel. Telescope images also consist of multiple filter bands, each absorbing photons within a specified range of wavelengths. 

For a given $H \times W$ pixel image with $B$ filter bands, our goal is to infer a catalog of 
stars. 
The catalog specifies the number of stars 
in the image; then for each star, the catalog 
records its location and its flux, or brightness,
in each band. 
The space of latent variables 
$\mathcal{Z}$ is the collection of all possible catalogs of the form
\begin{align}
    z := \{N, (\ell_i, f_{i,1}, ..., f_{i,B})_{i = 1}^N\},
    \label{eq:cat_formulation}
\end{align}
where the number of stars in the catalog
is $N\in\mathbb{N}$,
the location of the $i$th star is $\ell_i \in \Reals^2$, and 
the flux of the $i$th star in the $b$th band is $f_{i, b}\in\Reals^+$. 

A Bayesian approach requires specification of a prior over catalog space $\mathcal{Z}$ and a likelihood for the observed images. Our generative model, detailed below, is similar to previous approaches in~\cite{Brewer_2013, Portillo_2017, Feder_2019, regier2019_celeste}. 

\subsection{The prior}
The prior over $\mathcal{Z}$ is marked spatial Poisson process. First, draw the number of stars contained in the $H\times W$ image as
\begin{align}
	N &\sim \text{Poisson}(\mu HW).
	\label{eq:n_prior}
\end{align}
% \jeff{Better to talk about the marked point process later (or earlier, in the introduction). Just describe the model as simply as possible here.}
Next, draw locations
\begin{align}
  \ell_1, ..., \ell_N | N &\stackrel{iid}{\sim} \text{Uniform}([0, H] \times [0, W]). 
 \end{align}
The fluxes in the first band are from a power law distribution:
\begin{align}
    f_{1, 1}, ..., f_{N,1} | N & 
    \stackrel{iid}{\sim} \text{Pareto}(f_{min}, \alpha) 
    \label{eq:flux_prior}.
\end{align}
Fluxes in other bands are described relative to the first band. The log-ratio of flux relative to the first band is
known as color. Colors are drawn from a Gaussian distribution
\begin{align}
  c_{1, b}, ..., c_{N,b} | N  & 
      \stackrel{iid}{\sim} \mathcal{N}(\mu_c, \sigma^2_c), \quad b = 2, ..., B.
\end{align}
Given the flux in the first band $f_{i,1}$ and color $c_{i,b}$,
the flux in band $b$ is  $f_{i,b} = f_{i,1} \times 10^{c_{i,b} / 2.5}$.

\subsection{The likelihood}
Let $x_{hw}^b$ denote the observed number of photoelectrons at pixel $(h,w)$ and band $b$. For each band, at every pixel, the expected number of photoelectron arrivals is $\lambda^b_{hw}(z)$, a deterministic function of the catalog $z$. Motivated by the Poissonian nature of photon arrivals, the observations are drawn
\begin{align}
  x_{hw}^b | z \overset{ind}{\sim} \mathcal{N}(\lambda^b_{hw}, \lambda^b_{hw}),
  \quad 
  b = 1, ..., B; \;
  h = 1,..., H; \; 
  w = 1, ..., W. 
\end{align}
where 
\begin{align}
  \lambda^b_{hw} = I^{b}(h, w) + \sum_{i = 1}^N f_{i,b} \mathcal{P}^b\big(h - \ell_{i}[1], w - \ell_{i}[2]\big).
  \label{eq:expected_intensity}
\end{align}
Here, $\mathcal{P}^b$ is the point spread function (PSF) for band $b$ and $I^{b}$ is the background intensity. The PSF is a function 
\begin{align}
\mathcal{P}^b : \Reals \times \Reals \mapsto \Reals^+,
\end{align}
describing how a stellar point source appears at any 2D position of the image. The PSF model is a weighted average between a Gaussian ``core" and a power-law ``wing" as described in~\cite{Xin2018psf}. For each band, the PSF has the form
\jeff{Better to define the parameters before using them.}
\begin{align}
    \mathcal{P}(h, w) = \frac{\exp(\frac{-(h^2 + w^2)}{2\sigma_1^2}) + 
                            \zeta \exp(\frac{-(h^2 + w^2)}{2\sigma_2^2}) + 
                            \rho(1 + \frac{h^2 + w^2}{\gamma\sigma^2_P})^{-\gamma/2} }{1 + \zeta + \rho},
\end{align}
The PSF parameters vary by band. Let the collection of PSF parameters across all bands be denoted $\pi := (\sigma_{1}^{(b)}, \sigma_{2}^{(b)}, \sigma_{P}^{(b)}, \gamma^{(b)}, \zeta^{(b)}, \rho^{(b)})_{b=1}^B$.

The background intensity at pixel $(h,w)$ is modeled with an affine function: 
\begin{align}
    I^{b}(h,w) = \beta_0^{b} + \beta_1^{b} \times h + \beta_2^{b} \times w.
\end{align}
The background parameters are specific to the band. 

The model parameters $\pi$ and $\beta$ are estimated by the SDSS software pipeline and reported along with the release of each SDSS image.
Prior work on probabilistic cataloging relied 
on these SDSS estimates and found the PSF estimates to be suboptimal in crowded starfields~\cite{Feder_2019}. 
In contrast, StarNet estimates these parameters jointly with the approximate posterior (Section~\ref{sec:wake_sleep}). 

