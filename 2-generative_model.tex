For a given $H \times W$ pixel image, our goal is to infer a catalog of 
stars; for each star, the catalog consists of a location, brightness, and color. Our space of latent variables 
$\Theta$ is the collection of all possible catalogs of the form
\begin{align}
    \theta := \{N, (\ell_i, f_{i,1}, ..., f_{i,B})_{i = 1}^N\},
    \label{eq:cat_formulation}
\end{align}
where the number of stars in the catalog
is $N\in\mathbb{N}$;
\jeff{Unfortunately I don't think astronomers define catalog this way. They think of a catalog as including uncertainty estimates, not as a particularly realization of some random variables.}
\bryan{I think this is actually how Portillos refers to a catalog -- he does admit that this isn't 
what astronomers traditionally think of a catalog. The reason we need this is because our posterior is really defined over the space of all possible catalogs -- each sample from the posterior is a catalog as defined in \eqref{eq:cat_formulation}. I'll try to set this up more explicitly in the introduction. }
the location of the $i$th star is $\ell_i \in \Reals^2$;
and the flux in the $b$th image band of the $i$th star is $f_{i, b}\in\Reals^+$. 

Our prior is a distribution over the set of catalogs $\Theta$ is defined by a marked spatial Poisson process. That is, we first draw the number of stars contained in the $H\times W$ image:
\begin{align}
	N &\sim \text{Poisson}(\mu HW).
\end{align}
% \jeff{Better to talk about the marked point process later (or earlier, in the introduction). Just describe the model as simply as possible here.}
Next, we draw locations
\begin{align}
  \ell_{i} &\sim \text{Uniform}([0, H] \times [0, W]) \quad \forall i = 1, ..., N. 
 \end{align}
We draw fluxes in the first band from a power law distribution:
\begin{align}
    f_{i, 1} & \sim \text{Pareto}(f_{min}, \alpha) \quad \forall i = 1, ..., N. 
\label{eq:flux_prior}
\end{align}
We then draw colors,
\begin{align}
  c_{i, b}  & \sim \mathcal{N}(\mu_c, \sigma^2_c) \quad \forall b = 2, ..., B; i = 1, ..., N.
\end{align}
We call the first band the ``reference band" because fluxes in bands $2, ..., B$ are defined with respect to the first band. Given flux in the reference band $f_{i,1}$ and color $c_{i,b}$,
the flux in the $b$th band is  $f_{i,b} = f_{i,1} \times 10^{c_{i,b} / 2.5}$.
\jeff{Is the reference band always 1? I usually think of the r band as having index 3. What does Portillo do?}
\bryan{I didn't actually mean the index of the band in the telescope -- just wanted an enumerate of bands $1, ..., B$}
% the flux in the $b$th band $f_{i,b}$ is $f_{i,1} \times 10^{c_{i,b} / 2.5}$.
% \jeff{$f^b$ looks like a value raised to an exponent. $f^{(b)}$ instead?}

% We call the first band a ``reference band'' and define colors with respect to the band. 
% An image with $B$ bands, has $B - 1$ colors.
% The relation between flux in the $b$th band $f_b$ and color $c_{i, b}$ is 
% $c_{i,b} = 2.5\log_10(f_{i,b}/f_{i,1})$

% In our formulation, each value of $N$ corresponds to a disjoint set of $N$ locations and fluxes.
% Let $\ell$ and $f$ (without subscripts) each denote the triangular array of latent variables.

% Having drawn $N$, we index into the $N$th row of triangular array, and use this set of locations and fluxes to construct 
% the image. The expected number of photoelectrons measured at pixel $(h,w)$ in band $b$ is

Having drawn a catalog $\theta = \{N, (\ell_i, f_{i,1}, ..., f_{i,B})_{i = 1}^N\}$,
the expected number of photoelectrons measured at pixel $(h,w)$ in band $b$ is
\begin{align}
  \lambda^b_{hw} = I^{b}_{hw} + \sum_{i = 1}^N f_{i,b} \mathcal{P}^b\big(h - \ell_{i}[1], w - \ell_{i}[2]\big),
  \label{eq:expected_intensity}
\end{align}
where $I^{b}_{hw}$ is the background intensity, which we allow to vary by pixel and band. $\mathcal{P}^b$ is the point spread function (PSF) for band $b$. The PSF
is a function 
\begin{align}
\mathcal{P}^b : \Reals \times \Reals \mapsto \Reals^+,
\end{align}
describing how a stellar point source appears
on our image. We model the PSF as a weighted average between a Gaussian ``core" and a power-law ``wing" as described in~\cite{Xin2018psf}. For each band, the PSF has the form
\begin{align}
    \mathcal{P}(u,w) = \frac{\exp(\frac{-r^2}{2\sigma_1^2}) + 
                            b \exp(\frac{-r^2}{2\sigma_2^2}) + 
                            p_0(1 + \frac{r^2}{\gamma\sigma^2_P})^{-\beta/2} }{1 + b + p_0},
\end{align}
where $r^2 = u^2 + w^2$. The PSF parameters vary by band. Define 
$\pi := (\sigma_{1,b}, \sigma_{2,b}, \sigma_{P,b}, \gamma_b, p_{0,b})_{b=1}^B$ as the collection of PSF parameters across all bands. 

The background intensity is modeled as with an affine function: 
\begin{align}
    I_{ij}^{b} = \beta_0^{b} + \beta_1^{b} \times i + \beta_2^{b} \times j.
\end{align}
The background parameters are specific to the band. 

The model parameters $\pi$ for the PSF and $\beta$ are estimated by the SDSS software pipeline and reported along with the release of each SDSS image; in Section~\ref{sec:model_params}, we propose a method to estimate these parameters jointly with the catalog. 

Finally, we model the observed number of photoelectrons at pixel $(h,w)$ and band $b$ as Gaussian
with mean and variance equal to $\lambda^b_{hw}$. This model is reasonable due to the law of rare events.
% \jeff{If we're actually using a Gaussian rather than a Poisson, it's probably better not to mention the Poisson.
% Just say we model the number as Gaussian with var == mean.}
Thus, the observed pixel intensities are
\begin{align}
  x_{hw}^b | \theta \overset{ind}{\sim} \mathcal{N}(\lambda^b_{hw}, \lambda^b_{hw})
  \quad\forall b = 1, ..., B; h = 1,..., H; w = 1, ..., W. 
\end{align}
% \jeff{Better to use lowercase $x$ rather than $X$ if you're not going to use uppercase for all the random variables}
Explicitly, the log-likelihood is
\begin{align}
    \log p(x | \theta) &= \sum_{b = 1}^{B} \sum_{h = 1}^H \sum_{w = 1}^W 
        \Big\{\frac{1}{2\lambda^b_{hw}}(x_{hw}^b  - \lambda^b_{hw})^2 - 
               \frac{1}{2}\log(2\pi\lambda^b_{hw})\Big\}
    \label{eq:loglik}.
\end{align}

The likelihood and priors described above are similar to those
presented in~\cite{Portillo_2017, Feder_2019}. We set $\alpha = 2$; we set 
$f_{min}$ to correspond with the SDSS 
lower detection limit ($\approx$ 22 magnitude); we use a standard Gaussian for color priors. 

One difference is that \cite{Portillo_2017, Feder_2019} use an exponential prior on $N$. We use a Poisson prior and set $\mu = 0.15$, slightly denser than the $0.1$ stars per pixel 
brighter than 22mag found in the Hubble image of M2. 

In summary, we have detailed the likelihood
$p(x | \theta)$ which describes the generation of an image $x$ given a catalog $\theta$. We have specified a prior distribution $p(\theta)$ over the space 
of all catalogs $\Theta$. 
Our goal is to answer the question, ``given observed image $x$, which 
catalogs best explain the data?"
The question is answered by the posterior distribution $p(\theta | x)$. While 
the exact posterior distribution is intractable, we detail our variational approximation to the posterior distribution in the next section. 