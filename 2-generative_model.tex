In crowded starfields, such as globular clusters and the Milky Way, the vast majority of light sources are stars.
An astronomical image records the number of photons that reached a telescope and arrived at each pixel. 
Typically, photons must pass through one of several filters, each selecting photons from a specified band of wavelengths, before being recorded.

For a given $H \times W$ pixel image with $B$ filter bands, our goal is to infer a catalog of 
stars. 
The catalog specifies the number of stars 
in an image; for each such star, the catalog 
records its location and its flux, or brightness,
in each band. 
The space of latent variables 
$\mathcal{Z}$ is the collection of all possible catalogs of the form
\begin{align}
    z := \{N, (\ell_i, f_{i,1}, ..., f_{i,B})_{i = 1}^N\},
    \label{eq:cat_formulation}
\end{align}
where the number of stars in the catalog
is $N\in\mathbb{N}$,
the location of the $i$th star is $\ell_i \in \Reals^2$, and 
the flux of the $i$th star in the $b$th band is $f_{i, b}\in\Reals^+$. 

A Bayesian approach requires specification of a prior over catalog space $\mathcal{Z}$ and a likelihood for the observed images. Our likelihood and prior, detailed below, are similar to previous approaches~\cite{Portillo_2017, Brewer_2013, Feder_2019, regier2019_celeste}.

\subsection{The prior}
The prior over $\mathcal{Z}$ is a marked spatial Poisson process. To sample the prior, first draw the number of stars contained in the $H\times W$ image as
\begin{align}
	N &\sim \text{Poisson}(\mu HW),
	\label{eq:n_prior}
\end{align}
where $\mu$ is a hyperparameter specifying the average number of sources per pixel.
Next, draw locations
\begin{align}
  \ell_1, ..., \ell_N | N &\stackrel{iid}{\sim} \text{Uniform}([0, H] \times [0, W]). 
 \end{align}
The fluxes in the first band are from a power law distribution with slope $\alpha$:
\begin{align}
    f_{1, 1}, ..., f_{N,1} | N & 
    \stackrel{iid}{\sim} \text{Pareto}(f_{min}, \alpha) 
    \label{eq:flux_prior}.
\end{align}
Fluxes in other bands are described relative to the first band. Like~\cite{Feder_2019}, we define the log-ratio of flux relative to the first band as ``color." Colors are drawn from a Gaussian distribution
\begin{align}
  c_{1, b}, ..., c_{N,b} | N  & 
      \stackrel{iid}{\sim} \mathcal{N}(\mu_c, \sigma^2_c), \quad b = 2, ..., B.
\end{align}
Given the flux in the first band $f_{i,1}$ and color $c_{i,b}$,
the flux in band $b$ is  $f_{i,b} = f_{i,1} \times 10^{c_{i,b} / 2.5}$.

Also like~\cite{Feder_2019}, we set the power law slope $\alpha = 0.5$ and use a standard Gaussian for the color prior ($\mu_c = 0$, $\sigma^2_c = 1$). 
Appendix~\ref{sec:prior_sensitivity} evaluates the 
sensitivity of the resulting catalog to choices of the prior parameters. 

\subsection{The likelihood}
Let $x_{hw}^b$ denote the observed number of photoelectrons at pixel $(h,w)$ in band $b$. 
For each band, at every pixel, the expected number of photoelectron arrivals is $\lambda^b_{hw}(z)$, a deterministic function of the catalog $z$. Motivated by the Poissonian nature of photon arrivals and 
the large photon arrival rate in SDSS and LSST images, 
observations are drawn as
\begin{align}
  x_{hw}^b | z \overset{ind}{\sim} \mathcal{N}(\lambda^b_{hw}, \lambda^b_{hw}),
  \quad 
  b = 1, ..., B; \;
  h = 1,..., H; \; 
  w = 1, ..., W, 
\end{align}
where 
\begin{align}
  \lambda^b_{hw} = I^{b}(h, w) + \sum_{i = 1}^N f_{i,b} \mathcal{P}^b\big(h - \ell_{i, 1}, w - \ell_{i, 2}\big).
  \label{eq:expected_intensity}
\end{align}
Here, $\mathcal{P}^b$ is the point spread function (PSF) for band $b$ and $I^{b}$ is the background intensity. The PSF is a function 
\begin{align}
\mathcal{P}^b : \Reals \times \Reals \mapsto \Reals^+,
\end{align}
describing the appearance of a stellar point source at any 2D position of the image (but ignoring pixelation). 
Our PSF model is a weighted average between a Gaussian ``core" and a power-law ``wing," as described in~\cite{Xin2018psf}. For each band, the PSF has the form

\begin{align}
    \mathcal{P}(u,v) = 
    \frac{\exp(\frac{-(u^2 + v^2)}{2\sigma_1^2}) + 
    \zeta \exp(\frac{-(u^2 + v^2)}{2\sigma_2^2}) + 
    \rho(1 + \frac{v^2 + u^2}{\gamma\sigma^2_P})^{-\gamma/2} }{1 + \zeta + \rho}.
\end{align}
The PSF parameters vary by band. Let the collection of PSF parameters across all bands be denoted $\pi := (\sigma_{1}^{(b)}, \sigma_{2}^{(b)}, \sigma_{P}^{(b)}, \gamma^{(b)}, \zeta^{(b)}, \rho^{(b)})_{b=1}^B$.

The background intensity at pixel $(h,w)$ is modeled with an affine function: 
\begin{align}
    I^{b}(h,w) = \beta_0^{b} + \beta_1^{b} \times h + \beta_2^{b} \times w.
\end{align}
The background parameters are specific to the band. 

StarNet estimates these parameters jointly with the approximate posterior (Section~\ref{sec:wake_sleep}). 
Prior work on probabilistic cataloging relied 
on estimates from the SDSS software pipeline and found the PSF estimates to be suboptimal in crowded starfields~\cite{Feder_2019}. 


