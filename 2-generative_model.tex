To generate a $H \times W$ pixel image, we first draw the number of stars it contains:
\begin{align}
	N &\sim \text{Poisson}(\mu HW),
\end{align}
\jeff{Consider making $N$ lowercase, so that all our random variables are lowercase. It might look weird at first, but it's really helpful to distinguish with notation what is random and what isn't.}
\jeff{Better to talk about the marked point process later (or earlier, in the introduction). Just describe the model as simply as possible here.}
Then, for each $N = 1, 2, ... $, we draw locations 
\begin{align}
  \ell_{N, i} &\sim \text{Uniform}([0, H] \times [0, W]) \quad \forall i = 1, ..., N, 
 \end{align}
 and reference band fluxes
 \begin{align}
  f_{N, i}^{1} & \sim \text{Pareto}(f_{min}, \alpha) \quad \forall i = 1, ..., N. 
  \label{eq:flux_prior}
\end{align}
Images have $B$ bands and therefore $B - 1$ colors. The color distribution is
\begin{align}
  c_{N, i}^{b}  & \sim \mathcal{N}(\mu_c, \sigma^2_c) \quad \forall b = 2, ..., B; i = 1, ..., N.
\end{align}
The relation between flux and color is as follows: 
given flux in the reference band $f^1$ and color $c^b$,
\jeff{Is the reference band always 1? I usually think of the r band as having index 3. What does Portillo do?}
the flux in the $b$th band $f^b$ is $f^1 \times 10^{c^b / 2.5}$.
\jeff{$f^b$ looks like a value raised to an exponent. $f^{(b)}$ instead?}
In the SDSS data, we took the $r$-band to be the reference band.

Each value of $N$ corresponds to a disjoint set of $N$ locations and fluxes.
Let $\ell$ and $f$ (without subscripts) each denote the triangular array of latent variables.

Given $N$, $\ell$, and $f$, the expected number of photoelectrons measured at pixel $(h,w)$ in band $b$ is
\begin{align}
  \lambda^b_{hw} = I^{b}_{hw} + \sum_{i = 1}^N f_{N, i}^b \mathcal{P}^b\big(h - \ell_{N, i}[1], w - \ell_{N,i}[2]\big),
  \label{eq:expected_intensity}
\end{align}
where $I^{b}_{hw}$ is the background intensity, which we allow to vary by pixel and band,
and where $\mathcal{P}^b$ is the point spread function (PSF) for band $b$. The PSF
is a function
\begin{align}
\mathcal{P}^b : [0, H] \times [0, W] \mapsto \Reals^+,
\end{align}
\jeff{Is this the same $H$ as appears earlier?}
describing how a stellar point source appears
on our image. In Section~\ref{sec:model_params}, we will see 
that the model
parameters $I^{b}_{hw}$ and $\mathcal{P}^b$
\jeff{The PSF is a function, not model parameters. What are the parameters the PSF depends on? Those are part of the generative model.}
can be estimated jointly with
the latent variables.

Making an appeal to the law of rare events, we model the
observed number of photoelectrons at pixel $(h,w)$ and band $b$ as Poisson
with mean $\lambda^b_{hw}$. Since $\lambda^b_{hw}$ is large,
we take the Gaussian approximation to the Poisson.
Thus, the observed pixel intensities are
\begin{align}
  X_{hw}^b | f, \ell, N \overset{ind}{\sim} \mathcal{N}(\lambda^b_{hw}, \lambda^b_{hw}).
\end{align}
\jeff{Better to use lowercase $x$ rather than $X$ if you're not going to use uppercase for all the random variables}
Explicitly, the log-likelihood is
\begin{align}
    \log P(X | N, \ell, f) &= \sum_{b = 1}^{B} \sum_{h = 1}^H \sum_{w = 1}^W 
        \Big\{\frac{1}{2\lambda^b_{hw}}(X_{hw}^b  - \lambda^b_{hw})^2 - 
               \frac{1}{2}\log(2\pi\lambda^b_{hw})\Big\}
    \label{eq:loglik}.
\end{align}