In images collected by astronomical surveys, stars and galaxies often overlap visually.
Deblending is the task of distinguishing and characterizing individual light sources in survey images.
We propose StarNet, a fully Bayesian method to deblend sources in astronomical images of crowded star fields.
StarNet leverages recent advances in variational inference,
including amortized variational distributions and
an optimization objective targeting an expectation of the forward KL divergence.
% Wake-sleep, which minimizes forward KL divergence,
% has significant benefits compared to traditional variational inference,
% which minimizes a reverse KL divergence.
In our experiments with SDSS images of the M2 globular cluster, StarNet is substantially more accurate than two competing methods: Probablistic Cataloging (PCAT), a method that uses MCMC for inference, and DAOPHOT, a software pipeline employed by SDSS for deblending.
In addition, the amortized approach to inference 
gives StarNet the scaling characteristics necessary to perform fully Bayesian inference on modern astronomical surveys.

% In addition, StarNet is as much as $100$ times faster than PCAT, exhibiting the scaling characteristics necessary to perform fully Bayesian inference on modern astronomical surveys.
