% \subsection{Variational EM}
% The wake-sleep algorithm is closely related to variational EM~\citep{Jordan_intro_vi, neal2000varem, Beal2002varem},
% which alternates between an {\itshape expectation} step (E-step) and a {\itshape maximization} step (M-step).
% Variational EM can be viewed as block coordinate ascent on the ELBO objective, with the E-step optimizing variational parameters $\eta$ and the M-step optimizing the model parameters $\phi$.
% The wake phase of the wake-sleep algorithm is equivalent to the M-step of variational EM.
% As discussed above, the optimization of the ELBO with respect to $\eta$ is not conducive to using simple stochastic gradient estimators.
% Thus, the sleep phase replaces the ELBO objective of the E-step with the expected KL in~\eqref{eq:sleep_obj}.


% \begin{align}
%     \text{{\bf E-step: }} &
%     \eta_{t} = \argmax_{\eta}\; \Expect_{q_\eta(z | x)}\Big[\log p_{\phi_{t-1}}(x, z) - \log q_\eta(z | x)\Big];
%     \label{eq:e_step}
%     \\
%     \text{{\bf M-step: }} & \phi_{t} = \argmax_{\phi}\; \Expect_{q_{\eta_t}(z | x)}\Big[\log p_{\phi}(x, z) - \log q_{\eta_t}(z | x)\Big],
%     \label{eq:m_step}
% \end{align}
% for iterations $t = 1, ..., T$.

% Variational EM can be viewed as block coordinate ascent on the ELBO objective, with the E-step optimizing variational parameters $\eta$ and the M-step optimizing the model parameters $\phi$.

% The wake phase of the wake-sleep algorithm is equivalent to the M-step of variational EM.
% As discussed above, the optimization of variational parameters $\eta$ using the ELBO objective is not conducive to using simple stochastic gradient estimators.
% Thus, the sleep phase replaces the ELBO objective of the E-step with the expected reverse KL in~\eqref{eq:sleep_obj}.
