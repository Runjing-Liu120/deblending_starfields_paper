The E-step objective~\eqref{eq:e_step} is of the form 
\begin{align}
    \mathcal{L}(\eta) = \Expect_{q_\eta(z)}[f_\eta(z)]
\end{align}
where $\eta$ is the parameter to be optimized and $z$ is the latent variable. The integrating distribution $q$ and the function $f$ depend on $\eta$. 

The REINFORCE~\cite{Williams1992reinforce}, or score function estimator, is a general-purpose unbiased estimate for the gradient $\nabla_\eta \mathcal{L}(\eta)$. It is given by 
\begin{align}
    g_{\textrm{rf}}(z) = \nabla_\eta f_\eta(z) + 
            f_\eta(z)  \nabla_\eta \log q_\eta(z)
    \quad \text{for } 
    z\sim q_\eta(z). 
\end{align}
The REINFORCE estimate is unbiased for the true gradient, \begin{align}
    \Expect_{q_\eta(z)}[g_{\textrm{rf}}(z)] &= 
    \int q_\eta(z) \nabla_\eta f_\eta(z) \; dz+ 
    \int q_\eta(z) f_\eta(z)  \nabla_\eta \log q_\eta(z)\; dz \\
    &= \int q_\eta(z) \nabla_\eta f_\eta(z) \; dz+ 
    \int f_\eta(z) \nabla_\eta q_\eta(z)  \; dz \\
    &= \int \nabla_\eta[q_\eta(z) f_\eta(z)] \; dz \\
    &= \nabla_\eta \int q_\eta(z) f_\eta(z) \; dz 
    = \nabla_\eta \Expect_{q_\eta(z)}[f_\eta(z)],
\end{align}
assuming that $f$ is well-behaved so that integration and differentiation can be interchanged. 

Alternatively, the reparameterized gradient~\cite{rezende2014stochastic, kingma2013autoencoding} can be used when there exists some distribution $F$ independent of $\eta$, and a differentiable mapping $h_\eta$ such that 
\begin{align}
    w \sim F \implies h_\eta(w) \stackrel{d}{=} z,
\end{align}
with $z\sim q_\eta$. 

For example, if $q_{\eta}(z) = \mathcal{N}(z; \eta, 1)$ that is, a Gaussian with unit variance and mean $\eta$, one possibility is to let $F$ be the standard Gaussian and $h_\eta(w) = w + \eta$. 

The gradient of $\mathcal{L}(\eta)$ can be written as 
\begin{align}
    \nabla_\eta \Expect_{q_\eta(z)}[f_\eta(z)] &= 
        \nabla_\eta \Expect_{w\sim F}[f_\eta(h_\eta(w)] = \Expect_{w\sim F}[\nabla_\eta f_\eta(h_\eta(w)], 
\end{align}
again assuming the interchangability of integrals and derivatives. 

Unbiased gradients are computed as 
\begin{align}
    g_{\textrm{rp}} 
    = \nabla_\eta f_\eta(h_\eta(w)
    = \nabla_z f_\eta(z)\Big|_{z = h_\eta(w)}
    \nabla_\eta h_\eta(w) \quad \text{for } w\sim F. 
\end{align}

The reparameterized gradient includes gradient information $\nabla_z f_\eta(z)$ while the REINFORCE gradient does not. Taking into account the structure of $f$ through its gradient lowers the variance of reparameterized gradient in comparison to the REINFORCE gradient. 

However, if $z$ contains discrete components, there cannot not a differentiable mapping $h_\eta$, and the reparameterization trick will not apply. 

In our case (Section~\ref{sec:elbo_sleep_compare}), we used a combination of reparameterized and REINFORCE gradients. Let $N$ be the discrete component of $z$ (the number of stars) and $y$ be the continuous components (the locations and fluxes). Our variational distribution factorizes, so we write the expectation as 
\begin{align}
 \Expect_{q_\eta(N)}\Expect_{q_\eta(y)}[f_\eta(N, y)]. 
\end{align}
We use the REINFORCE estimator for the outer expectation and the reparameterization trick for the inner expectation. We first apply the REINFORCE to the outer expectation: 
\begin{align}
    \nabla_\eta  \Expect_{q_\eta(N)}&\Expect_{q_\eta(y)}\Big[f_\eta(N, y)\Big] \\
    &=  \Expect_{q_\eta(N)}\Big[ \nabla_\eta \log q_\eta(N) \Expect_{q_\eta(y)}[f_\eta(N, y)] + 
    \nabla_\eta \Expect_{q_\eta(y)}[f_\eta(N, y)] \Big]\\
    &\approx \nabla_\eta \log q_\eta(N) \Expect_{q_\eta(y)}[f_\eta(N, y)] + 
    \nabla_\eta \Expect_{q_\eta(y)}[f_\eta(N, y)] \quad \text{for } N \sim q_\eta(N). 
    \label{eq:reinforce_partial}
\end{align}
Then we use the reparameterization trick for $y$, so 
\begin{align}
    \Expect_{q_\eta(y)}[f_\eta(N, y)] &\approx f_\eta(N, h_\eta(w))\\
    \nabla_\eta \Expect_{q_\eta(y)}[f_\eta(N, y)] &\approx  \nabla_y f_\eta(N, y)\Big|_{y = h_\eta(w)}
    \nabla_\eta h_\eta(w) 
    \label{eq:reparam_partial}
\end{align}
for $w \sim F$, where $h_\eta$ and $F$ are chosen appropriately. Combining~\eqref{eq:reinforce_partial} and~\eqref{eq:reparam_partial}, our gradient estimator is 
\begin{align}
    g(z) = \nabla_\eta \log q_\eta(N) 
    f_\eta(N, h_\eta(w)) + 
    \nabla_y f_\eta(N, y)\Big|_{y = h_\eta(w)}
    \nabla_\eta h_\eta(w) 
\end{align}
for $N\sim q_\eta(N)$ and $w\sim F$. Because the variational distribution on locations and fluxes are transformation of Gaussians (recall locations are logit-normal and fluxes log-normal), we take $F$ to be the standard Gaussian, while $h(\cdot ; \eta)$ shifts and scales the standard Gaussian according the the parameters returned by the neural network and applies the appropriate transformation. 
