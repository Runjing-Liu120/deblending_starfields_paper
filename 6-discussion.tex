Probabilistic cataloging provides a framework to 
produce catalogs with statistically principled uncertainties. We defined a statistical model, and uncertainties are captured by a Bayesian posterior over the space of all catalogs.
In previous work on probabilistic cataloging, samples from the posterior were obtained using MCMC. While samples eventually converge to the exact posterior, the computational cost of MCMC prohibits its application to large-scale astronomical surveys. 

In this work, we produced an approximate Bayesian posterior using amortized variational inference, which has potential to scale Bayesian inference to large astronomical surveys. We employed neural networks to efficiently return approximate posteriors for a given image. We chose to train the neural network using the wake-sleep algorithm which optimizes an objective different than the usual ELBO. We found that in this application, the wake-sleep algorithm was able to avoid the local minima present in the ELBO objective. 

In addition to doing inference, we also estimated model parameters such as the PSF and sky background. While the current work focuses on PSF models, our method can be extended to more general sources such as galaxies.
In addition to location, flux, and color, each source would also have shape parameters. One could assume a known parametric model for a galaxy, such as exponential or de-Vaucouleurs profile. 

Alternatively, one could fit a deep generative model for galaxies~\cite{Regier2015ADG} using the wake-sleep algorithm. Here, a neural network takes as input low-dimensional source parameters and outputs a galaxy image. Neural networks would be trained in both the sleep and wake phase -- the sleep phase trains the approximate posterior while the wake phase trains the galaxy model. We believe that our training procedure. 

We believe that this training procedure holds great promise