Probabilistic cataloging provides a framework to 
produce catalogs with statistically principled uncertainties. We defined a statistical model, and uncertainties are captured by a Bayesian posterior over the space of all catalogs. 
These uncertainties are especially important in the cataloging of crowded starfields, where many light sources are blended; there is ambiguity not only in the location and fluxes of individual light sources, but also in the number of sources. 

In previous work on probabilistic cataloging, samples from the posterior were obtained using MCMC. While MCMC eventually converges to the exact posterior, its computational cost prohibits its application to large-scale astronomical surveys. 

Instead, we applied variational inference, and constructed an approximate posterior. The approximate posterior was defined using a neural network. Once trained, the neural network can be cheaply evaluated on batches of images to produce probabilistic catalogs. 